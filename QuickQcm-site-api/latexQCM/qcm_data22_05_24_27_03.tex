\documentclass{book}%
\usepackage[T1]{fontenc}%
\usepackage[utf8]{inputenc}%
\usepackage{lmodern}%
\usepackage{textcomp}%
\usepackage{lastpage}%
\input{structure}%
\usepackage{enumitem}%
%
%
%
\begin{document}%
\normalsize%
\newpage%
\thispagestyle{empty}%
\vskip-40mm	\includegraphics[scale=0.5]{logo.png} \\%
 \begin{flushright}  \vskip-20mm   Professeur: Hatim Naqos\vskip15mm  \end{flushright}%
JM3%
 \begin{flushright}  \vskip-7mm2024-05-22 \end{flushright}%
\begin{center}   \begin{Large}test : MATHÉMATIQUE\end{Large} \end{center}%
Durée : 2h%
 \begin{center} { \large CONSIGNES SPÉCIFIQUES } \\\ Lisez soigneusement les consignes ci-dessous afin de réussir au mieux cette épreuve : \end{center} %
\begin{itemize}%
\item%
L'usage de la calculatrice ou de tout autre appareil électronique est interdit.%
\item%
Aucun autre document que ce sujet et sa grille réponse n'est autorisé.%
\item%
Pour chacune des questions, indiquez sur la feuille de réponses ci-jointes, si les affirmations A, B, C et D sont (\textbf{V}) vraies ou (\textbf{F}) fausses en faisant une croix dans la colonne correspondant à votre choix. Vous ne pouvez pas faire de ratures. En cas d'erreur, utilisez la deuxième colonne de réponse. Si la deuxième colonne comporte au moins une réponse, la première colonne ne sera pas corrigée, c'est la deuxième qui sera prise en considération.%
\item%
Chaque réponse exacte est gratifiée de 2 points, tandis que chaque réponse fausse est pénalisée par -1 point. \\ 	Parmi les quatre propositions de chacune des questions \textbf{de 1 à 1}, une seule est vraie, les autres sont fausses. \\ 	Par exemple : Pour indiquer que l'affirmation $B$ est Vraie, cocher les cases comme suit:  \\ \begin{center}	\includegraphics[scale=0.8]{reponses.png} \end{center}%
\thispagestyle{empty}%
\begin{exercise}%
\textbf{test }%
\begin{enumerate}[label=\textbf{\Alph*. }]%
\item%
test%
\item%
test%
\item%
test%
\item%
test%
\end{enumerate}%
\end{exercise}%
\end{itemize}%
\newpage%
\thispagestyle{empty}%
test : MATHÉMATIQUE $\qquad \qquad \qquad \qquad \qquad \qquad \qquad \qquad$ ABIOLA Kolawole Japhet Martin%
\begin{flushright}%
\begin{tabular}{|l|}%
\hline%
 \\%
\thispagestyle{empty}%
Identifiant: $\quad$ {\Large 1~8~2~3~4~}%
 \\%
\hline%
\end{tabular}%
\end{flushright}%
\begin{center}%
\begin{tabular}{| l l l l l |}%
\hline%
 & & & & \\%
Question 1\qquad \qquad\ & & & & \\%
 & A $\qquad \square \qquad$ & B $\qquad \square \qquad$ & C $\qquad \square \qquad$ & D $\qquad \square \qquad$ \\ %
 & & & &  \\%
\hline%
 & & & &  \\%
Choix 2 & A $\qquad \square \qquad$ & B $\qquad \square \qquad$ & C $\qquad \square \qquad$ & D $\qquad \square \qquad$ \\ %
 & & & &  \\%
\hline%
\end{tabular}%
\\ \vskip3mm%
\thispagestyle{empty}%
\end{center}%
\newpage%
\thispagestyle{empty}%
\vskip-40mm	\includegraphics[scale=0.5]{logo.png} \\%
 \begin{flushright}  \vskip-20mm   Professeur: Hatim Naqos\vskip15mm  \end{flushright}%
JM3%
 \begin{flushright}  \vskip-7mm2024-05-22 \end{flushright}%
\begin{center}   \begin{Large}test : MATHÉMATIQUE\end{Large} \end{center}%
Durée : 2h%
 \begin{center} { \large CONSIGNES SPÉCIFIQUES } \\\ Lisez soigneusement les consignes ci-dessous afin de réussir au mieux cette épreuve : \end{center} %
\begin{itemize}%
\item%
L'usage de la calculatrice ou de tout autre appareil électronique est interdit.%
\item%
Aucun autre document que ce sujet et sa grille réponse n'est autorisé.%
\item%
Pour chacune des questions, indiquez sur la feuille de réponses ci-jointes, si les affirmations A, B, C et D sont (\textbf{V}) vraies ou (\textbf{F}) fausses en faisant une croix dans la colonne correspondant à votre choix. Vous ne pouvez pas faire de ratures. En cas d'erreur, utilisez la deuxième colonne de réponse. Si la deuxième colonne comporte au moins une réponse, la première colonne ne sera pas corrigée, c'est la deuxième qui sera prise en considération.%
\item%
Chaque réponse exacte est gratifiée de 2 points, tandis que chaque réponse fausse est pénalisée par -1 point. \\ 	Parmi les quatre propositions de chacune des questions \textbf{de 1 à 1}, une seule est vraie, les autres sont fausses. \\ 	Par exemple : Pour indiquer que l'affirmation $B$ est Vraie, cocher les cases comme suit:  \\ \begin{center}	\includegraphics[scale=0.8]{reponses.png} \end{center}%
\thispagestyle{empty}%
\begin{exercise}%
\textbf{test }%
\begin{enumerate}[label=\textbf{\Alph*. }]%
\item%
test%
\item%
test%
\item%
test%
\item%
test%
\end{enumerate}%
\end{exercise}%
\end{itemize}%
\newpage%
\thispagestyle{empty}%
test : MATHÉMATIQUE $\qquad \qquad \qquad \qquad \qquad \qquad \qquad \qquad$ ABOUCHADI Ahlam%
\begin{flushright}%
\begin{tabular}{|l|}%
\hline%
 \\%
\thispagestyle{empty}%
Identifiant: $\quad$ {\Large 2~9~5~6~7~}%
 \\%
\hline%
\end{tabular}%
\end{flushright}%
\begin{center}%
\begin{tabular}{| l l l l l |}%
\hline%
 & & & & \\%
Question 1\qquad \qquad\ & & & & \\%
 & A $\qquad \square \qquad$ & B $\qquad \square \qquad$ & C $\qquad \square \qquad$ & D $\qquad \square \qquad$ \\ %
 & & & &  \\%
\hline%
 & & & &  \\%
Choix 2 & A $\qquad \square \qquad$ & B $\qquad \square \qquad$ & C $\qquad \square \qquad$ & D $\qquad \square \qquad$ \\ %
 & & & &  \\%
\hline%
\end{tabular}%
\\ \vskip3mm%
\thispagestyle{empty}%
\end{center}%
\newpage%
\thispagestyle{empty}%
\vskip-40mm	\includegraphics[scale=0.5]{logo.png} \\%
 \begin{flushright}  \vskip-20mm   Professeur: Hatim Naqos\vskip15mm  \end{flushright}%
JM3%
 \begin{flushright}  \vskip-7mm2024-05-22 \end{flushright}%
\begin{center}   \begin{Large}test : MATHÉMATIQUE\end{Large} \end{center}%
Durée : 2h%
 \begin{center} { \large CONSIGNES SPÉCIFIQUES } \\\ Lisez soigneusement les consignes ci-dessous afin de réussir au mieux cette épreuve : \end{center} %
\begin{itemize}%
\item%
L'usage de la calculatrice ou de tout autre appareil électronique est interdit.%
\item%
Aucun autre document que ce sujet et sa grille réponse n'est autorisé.%
\item%
Pour chacune des questions, indiquez sur la feuille de réponses ci-jointes, si les affirmations A, B, C et D sont (\textbf{V}) vraies ou (\textbf{F}) fausses en faisant une croix dans la colonne correspondant à votre choix. Vous ne pouvez pas faire de ratures. En cas d'erreur, utilisez la deuxième colonne de réponse. Si la deuxième colonne comporte au moins une réponse, la première colonne ne sera pas corrigée, c'est la deuxième qui sera prise en considération.%
\item%
Chaque réponse exacte est gratifiée de 2 points, tandis que chaque réponse fausse est pénalisée par -1 point. \\ 	Parmi les quatre propositions de chacune des questions \textbf{de 1 à 1}, une seule est vraie, les autres sont fausses. \\ 	Par exemple : Pour indiquer que l'affirmation $B$ est Vraie, cocher les cases comme suit:  \\ \begin{center}	\includegraphics[scale=0.8]{reponses.png} \end{center}%
\thispagestyle{empty}%
\begin{exercise}%
\textbf{test }%
\begin{enumerate}[label=\textbf{\Alph*. }]%
\item%
test%
\item%
test%
\item%
test%
\item%
test%
\end{enumerate}%
\end{exercise}%
\end{itemize}%
\newpage%
\thispagestyle{empty}%
test : MATHÉMATIQUE $\qquad \qquad \qquad \qquad \qquad \qquad \qquad \qquad$ AKLAMAVO Jesugnon Toudonou Uriel%
\begin{flushright}%
\begin{tabular}{|l|}%
\hline%
 \\%
\thispagestyle{empty}%
Identifiant: $\quad$ {\Large 3~4~7~8~9~}%
 \\%
\hline%
\end{tabular}%
\end{flushright}%
\begin{center}%
\begin{tabular}{| l l l l l |}%
\hline%
 & & & & \\%
Question 1\qquad \qquad\ & & & & \\%
 & A $\qquad \square \qquad$ & B $\qquad \square \qquad$ & C $\qquad \square \qquad$ & D $\qquad \square \qquad$ \\ %
 & & & &  \\%
\hline%
 & & & &  \\%
Choix 2 & A $\qquad \square \qquad$ & B $\qquad \square \qquad$ & C $\qquad \square \qquad$ & D $\qquad \square \qquad$ \\ %
 & & & &  \\%
\hline%
\end{tabular}%
\\ \vskip3mm%
\thispagestyle{empty}%
\end{center}%
\newpage%
\thispagestyle{empty}%
\vskip-40mm	\includegraphics[scale=0.5]{logo.png} \\%
 \begin{flushright}  \vskip-20mm   Professeur: Hatim Naqos\vskip15mm  \end{flushright}%
JM3%
 \begin{flushright}  \vskip-7mm2024-05-22 \end{flushright}%
\begin{center}   \begin{Large}test : MATHÉMATIQUE\end{Large} \end{center}%
Durée : 2h%
 \begin{center} { \large CONSIGNES SPÉCIFIQUES } \\\ Lisez soigneusement les consignes ci-dessous afin de réussir au mieux cette épreuve : \end{center} %
\begin{itemize}%
\item%
L'usage de la calculatrice ou de tout autre appareil électronique est interdit.%
\item%
Aucun autre document que ce sujet et sa grille réponse n'est autorisé.%
\item%
Pour chacune des questions, indiquez sur la feuille de réponses ci-jointes, si les affirmations A, B, C et D sont (\textbf{V}) vraies ou (\textbf{F}) fausses en faisant une croix dans la colonne correspondant à votre choix. Vous ne pouvez pas faire de ratures. En cas d'erreur, utilisez la deuxième colonne de réponse. Si la deuxième colonne comporte au moins une réponse, la première colonne ne sera pas corrigée, c'est la deuxième qui sera prise en considération.%
\item%
Chaque réponse exacte est gratifiée de 2 points, tandis que chaque réponse fausse est pénalisée par -1 point. \\ 	Parmi les quatre propositions de chacune des questions \textbf{de 1 à 1}, une seule est vraie, les autres sont fausses. \\ 	Par exemple : Pour indiquer que l'affirmation $B$ est Vraie, cocher les cases comme suit:  \\ \begin{center}	\includegraphics[scale=0.8]{reponses.png} \end{center}%
\thispagestyle{empty}%
\begin{exercise}%
\textbf{test }%
\begin{enumerate}[label=\textbf{\Alph*. }]%
\item%
test%
\item%
test%
\item%
test%
\item%
test%
\end{enumerate}%
\end{exercise}%
\end{itemize}%
\newpage%
\thispagestyle{empty}%
test : MATHÉMATIQUE $\qquad \qquad \qquad \qquad \qquad \qquad \qquad \qquad$ AMADDAH Yassine%
\begin{flushright}%
\begin{tabular}{|l|}%
\hline%
 \\%
\thispagestyle{empty}%
Identifiant: $\quad$ {\Large 4~5~0~1~2~}%
 \\%
\hline%
\end{tabular}%
\end{flushright}%
\begin{center}%
\begin{tabular}{| l l l l l |}%
\hline%
 & & & & \\%
Question 1\qquad \qquad\ & & & & \\%
 & A $\qquad \square \qquad$ & B $\qquad \square \qquad$ & C $\qquad \square \qquad$ & D $\qquad \square \qquad$ \\ %
 & & & &  \\%
\hline%
 & & & &  \\%
Choix 2 & A $\qquad \square \qquad$ & B $\qquad \square \qquad$ & C $\qquad \square \qquad$ & D $\qquad \square \qquad$ \\ %
 & & & &  \\%
\hline%
\end{tabular}%
\\ \vskip3mm%
\thispagestyle{empty}%
\end{center}%
\newpage%
\thispagestyle{empty}%
\vskip-40mm	\includegraphics[scale=0.5]{logo.png} \\%
 \begin{flushright}  \vskip-20mm   Professeur: Hatim Naqos\vskip15mm  \end{flushright}%
JM3%
 \begin{flushright}  \vskip-7mm2024-05-22 \end{flushright}%
\begin{center}   \begin{Large}test : MATHÉMATIQUE\end{Large} \end{center}%
Durée : 2h%
 \begin{center} { \large CONSIGNES SPÉCIFIQUES } \\\ Lisez soigneusement les consignes ci-dessous afin de réussir au mieux cette épreuve : \end{center} %
\begin{itemize}%
\item%
L'usage de la calculatrice ou de tout autre appareil électronique est interdit.%
\item%
Aucun autre document que ce sujet et sa grille réponse n'est autorisé.%
\item%
Pour chacune des questions, indiquez sur la feuille de réponses ci-jointes, si les affirmations A, B, C et D sont (\textbf{V}) vraies ou (\textbf{F}) fausses en faisant une croix dans la colonne correspondant à votre choix. Vous ne pouvez pas faire de ratures. En cas d'erreur, utilisez la deuxième colonne de réponse. Si la deuxième colonne comporte au moins une réponse, la première colonne ne sera pas corrigée, c'est la deuxième qui sera prise en considération.%
\item%
Chaque réponse exacte est gratifiée de 2 points, tandis que chaque réponse fausse est pénalisée par -1 point. \\ 	Parmi les quatre propositions de chacune des questions \textbf{de 1 à 1}, une seule est vraie, les autres sont fausses. \\ 	Par exemple : Pour indiquer que l'affirmation $B$ est Vraie, cocher les cases comme suit:  \\ \begin{center}	\includegraphics[scale=0.8]{reponses.png} \end{center}%
\thispagestyle{empty}%
\begin{exercise}%
\textbf{test }%
\begin{enumerate}[label=\textbf{\Alph*. }]%
\item%
test%
\item%
test%
\item%
test%
\item%
test%
\end{enumerate}%
\end{exercise}%
\end{itemize}%
\newpage%
\thispagestyle{empty}%
test : MATHÉMATIQUE $\qquad \qquad \qquad \qquad \qquad \qquad \qquad \qquad$ BELKHETAB Kawtar%
\begin{flushright}%
\begin{tabular}{|l|}%
\hline%
 \\%
\thispagestyle{empty}%
Identifiant: $\quad$ {\Large 5~6~1~2~3~}%
 \\%
\hline%
\end{tabular}%
\end{flushright}%
\begin{center}%
\begin{tabular}{| l l l l l |}%
\hline%
 & & & & \\%
Question 1\qquad \qquad\ & & & & \\%
 & A $\qquad \square \qquad$ & B $\qquad \square \qquad$ & C $\qquad \square \qquad$ & D $\qquad \square \qquad$ \\ %
 & & & &  \\%
\hline%
 & & & &  \\%
Choix 2 & A $\qquad \square \qquad$ & B $\qquad \square \qquad$ & C $\qquad \square \qquad$ & D $\qquad \square \qquad$ \\ %
 & & & &  \\%
\hline%
\end{tabular}%
\\ \vskip3mm%
\thispagestyle{empty}%
\end{center}%
\newpage%
\thispagestyle{empty}%
\vskip-40mm	\includegraphics[scale=0.5]{logo.png} \\%
 \begin{flushright}  \vskip-20mm   Professeur: Hatim Naqos\vskip15mm  \end{flushright}%
JM3%
 \begin{flushright}  \vskip-7mm2024-05-22 \end{flushright}%
\begin{center}   \begin{Large}test : MATHÉMATIQUE\end{Large} \end{center}%
Durée : 2h%
 \begin{center} { \large CONSIGNES SPÉCIFIQUES } \\\ Lisez soigneusement les consignes ci-dessous afin de réussir au mieux cette épreuve : \end{center} %
\begin{itemize}%
\item%
L'usage de la calculatrice ou de tout autre appareil électronique est interdit.%
\item%
Aucun autre document que ce sujet et sa grille réponse n'est autorisé.%
\item%
Pour chacune des questions, indiquez sur la feuille de réponses ci-jointes, si les affirmations A, B, C et D sont (\textbf{V}) vraies ou (\textbf{F}) fausses en faisant une croix dans la colonne correspondant à votre choix. Vous ne pouvez pas faire de ratures. En cas d'erreur, utilisez la deuxième colonne de réponse. Si la deuxième colonne comporte au moins une réponse, la première colonne ne sera pas corrigée, c'est la deuxième qui sera prise en considération.%
\item%
Chaque réponse exacte est gratifiée de 2 points, tandis que chaque réponse fausse est pénalisée par -1 point. \\ 	Parmi les quatre propositions de chacune des questions \textbf{de 1 à 1}, une seule est vraie, les autres sont fausses. \\ 	Par exemple : Pour indiquer que l'affirmation $B$ est Vraie, cocher les cases comme suit:  \\ \begin{center}	\includegraphics[scale=0.8]{reponses.png} \end{center}%
\thispagestyle{empty}%
\begin{exercise}%
\textbf{test }%
\begin{enumerate}[label=\textbf{\Alph*. }]%
\item%
test%
\item%
test%
\item%
test%
\item%
test%
\end{enumerate}%
\end{exercise}%
\end{itemize}%
\newpage%
\thispagestyle{empty}%
test : MATHÉMATIQUE $\qquad \qquad \qquad \qquad \qquad \qquad \qquad \qquad$ BENTISSE salma%
\begin{flushright}%
\begin{tabular}{|l|}%
\hline%
 \\%
\thispagestyle{empty}%
Identifiant: $\quad$ {\Large 6~7~3~4~5~}%
 \\%
\hline%
\end{tabular}%
\end{flushright}%
\begin{center}%
\begin{tabular}{| l l l l l |}%
\hline%
 & & & & \\%
Question 1\qquad \qquad\ & & & & \\%
 & A $\qquad \square \qquad$ & B $\qquad \square \qquad$ & C $\qquad \square \qquad$ & D $\qquad \square \qquad$ \\ %
 & & & &  \\%
\hline%
 & & & &  \\%
Choix 2 & A $\qquad \square \qquad$ & B $\qquad \square \qquad$ & C $\qquad \square \qquad$ & D $\qquad \square \qquad$ \\ %
 & & & &  \\%
\hline%
\end{tabular}%
\\ \vskip3mm%
\thispagestyle{empty}%
\end{center}%
\newpage%
\thispagestyle{empty}%
\vskip-40mm	\includegraphics[scale=0.5]{logo.png} \\%
 \begin{flushright}  \vskip-20mm   Professeur: Hatim Naqos\vskip15mm  \end{flushright}%
JM3%
 \begin{flushright}  \vskip-7mm2024-05-22 \end{flushright}%
\begin{center}   \begin{Large}test : MATHÉMATIQUE\end{Large} \end{center}%
Durée : 2h%
 \begin{center} { \large CONSIGNES SPÉCIFIQUES } \\\ Lisez soigneusement les consignes ci-dessous afin de réussir au mieux cette épreuve : \end{center} %
\begin{itemize}%
\item%
L'usage de la calculatrice ou de tout autre appareil électronique est interdit.%
\item%
Aucun autre document que ce sujet et sa grille réponse n'est autorisé.%
\item%
Pour chacune des questions, indiquez sur la feuille de réponses ci-jointes, si les affirmations A, B, C et D sont (\textbf{V}) vraies ou (\textbf{F}) fausses en faisant une croix dans la colonne correspondant à votre choix. Vous ne pouvez pas faire de ratures. En cas d'erreur, utilisez la deuxième colonne de réponse. Si la deuxième colonne comporte au moins une réponse, la première colonne ne sera pas corrigée, c'est la deuxième qui sera prise en considération.%
\item%
Chaque réponse exacte est gratifiée de 2 points, tandis que chaque réponse fausse est pénalisée par -1 point. \\ 	Parmi les quatre propositions de chacune des questions \textbf{de 1 à 1}, une seule est vraie, les autres sont fausses. \\ 	Par exemple : Pour indiquer que l'affirmation $B$ est Vraie, cocher les cases comme suit:  \\ \begin{center}	\includegraphics[scale=0.8]{reponses.png} \end{center}%
\thispagestyle{empty}%
\begin{exercise}%
\textbf{test }%
\begin{enumerate}[label=\textbf{\Alph*. }]%
\item%
test%
\item%
test%
\item%
test%
\item%
test%
\end{enumerate}%
\end{exercise}%
\end{itemize}%
\newpage%
\thispagestyle{empty}%
test : MATHÉMATIQUE $\qquad \qquad \qquad \qquad \qquad \qquad \qquad \qquad$ BERNOUSSI Wahb%
\begin{flushright}%
\begin{tabular}{|l|}%
\hline%
 \\%
\thispagestyle{empty}%
Identifiant: $\quad$ {\Large 7~8~4~5~6~}%
 \\%
\hline%
\end{tabular}%
\end{flushright}%
\begin{center}%
\begin{tabular}{| l l l l l |}%
\hline%
 & & & & \\%
Question 1\qquad \qquad\ & & & & \\%
 & A $\qquad \square \qquad$ & B $\qquad \square \qquad$ & C $\qquad \square \qquad$ & D $\qquad \square \qquad$ \\ %
 & & & &  \\%
\hline%
 & & & &  \\%
Choix 2 & A $\qquad \square \qquad$ & B $\qquad \square \qquad$ & C $\qquad \square \qquad$ & D $\qquad \square \qquad$ \\ %
 & & & &  \\%
\hline%
\end{tabular}%
\\ \vskip3mm%
\thispagestyle{empty}%
\end{center}%
\newpage%
\thispagestyle{empty}%
\vskip-40mm	\includegraphics[scale=0.5]{logo.png} \\%
 \begin{flushright}  \vskip-20mm   Professeur: Hatim Naqos\vskip15mm  \end{flushright}%
JM3%
 \begin{flushright}  \vskip-7mm2024-05-22 \end{flushright}%
\begin{center}   \begin{Large}test : MATHÉMATIQUE\end{Large} \end{center}%
Durée : 2h%
 \begin{center} { \large CONSIGNES SPÉCIFIQUES } \\\ Lisez soigneusement les consignes ci-dessous afin de réussir au mieux cette épreuve : \end{center} %
\begin{itemize}%
\item%
L'usage de la calculatrice ou de tout autre appareil électronique est interdit.%
\item%
Aucun autre document que ce sujet et sa grille réponse n'est autorisé.%
\item%
Pour chacune des questions, indiquez sur la feuille de réponses ci-jointes, si les affirmations A, B, C et D sont (\textbf{V}) vraies ou (\textbf{F}) fausses en faisant une croix dans la colonne correspondant à votre choix. Vous ne pouvez pas faire de ratures. En cas d'erreur, utilisez la deuxième colonne de réponse. Si la deuxième colonne comporte au moins une réponse, la première colonne ne sera pas corrigée, c'est la deuxième qui sera prise en considération.%
\item%
Chaque réponse exacte est gratifiée de 2 points, tandis que chaque réponse fausse est pénalisée par -1 point. \\ 	Parmi les quatre propositions de chacune des questions \textbf{de 1 à 1}, une seule est vraie, les autres sont fausses. \\ 	Par exemple : Pour indiquer que l'affirmation $B$ est Vraie, cocher les cases comme suit:  \\ \begin{center}	\includegraphics[scale=0.8]{reponses.png} \end{center}%
\thispagestyle{empty}%
\begin{exercise}%
\textbf{test }%
\begin{enumerate}[label=\textbf{\Alph*. }]%
\item%
test%
\item%
test%
\item%
test%
\item%
test%
\end{enumerate}%
\end{exercise}%
\end{itemize}%
\newpage%
\thispagestyle{empty}%
test : MATHÉMATIQUE $\qquad \qquad \qquad \qquad \qquad \qquad \qquad \qquad$ DOUCH Anas%
\begin{flushright}%
\begin{tabular}{|l|}%
\hline%
 \\%
\thispagestyle{empty}%
Identifiant: $\quad$ {\Large 8~9~5~6~7~}%
 \\%
\hline%
\end{tabular}%
\end{flushright}%
\begin{center}%
\begin{tabular}{| l l l l l |}%
\hline%
 & & & & \\%
Question 1\qquad \qquad\ & & & & \\%
 & A $\qquad \square \qquad$ & B $\qquad \square \qquad$ & C $\qquad \square \qquad$ & D $\qquad \square \qquad$ \\ %
 & & & &  \\%
\hline%
 & & & &  \\%
Choix 2 & A $\qquad \square \qquad$ & B $\qquad \square \qquad$ & C $\qquad \square \qquad$ & D $\qquad \square \qquad$ \\ %
 & & & &  \\%
\hline%
\end{tabular}%
\\ \vskip3mm%
\thispagestyle{empty}%
\end{center}%
\newpage%
\thispagestyle{empty}%
\vskip-40mm	\includegraphics[scale=0.5]{logo.png} \\%
 \begin{flushright}  \vskip-20mm   Professeur: Hatim Naqos\vskip15mm  \end{flushright}%
JM3%
 \begin{flushright}  \vskip-7mm2024-05-22 \end{flushright}%
\begin{center}   \begin{Large}test : MATHÉMATIQUE\end{Large} \end{center}%
Durée : 2h%
 \begin{center} { \large CONSIGNES SPÉCIFIQUES } \\\ Lisez soigneusement les consignes ci-dessous afin de réussir au mieux cette épreuve : \end{center} %
\begin{itemize}%
\item%
L'usage de la calculatrice ou de tout autre appareil électronique est interdit.%
\item%
Aucun autre document que ce sujet et sa grille réponse n'est autorisé.%
\item%
Pour chacune des questions, indiquez sur la feuille de réponses ci-jointes, si les affirmations A, B, C et D sont (\textbf{V}) vraies ou (\textbf{F}) fausses en faisant une croix dans la colonne correspondant à votre choix. Vous ne pouvez pas faire de ratures. En cas d'erreur, utilisez la deuxième colonne de réponse. Si la deuxième colonne comporte au moins une réponse, la première colonne ne sera pas corrigée, c'est la deuxième qui sera prise en considération.%
\item%
Chaque réponse exacte est gratifiée de 2 points, tandis que chaque réponse fausse est pénalisée par -1 point. \\ 	Parmi les quatre propositions de chacune des questions \textbf{de 1 à 1}, une seule est vraie, les autres sont fausses. \\ 	Par exemple : Pour indiquer que l'affirmation $B$ est Vraie, cocher les cases comme suit:  \\ \begin{center}	\includegraphics[scale=0.8]{reponses.png} \end{center}%
\thispagestyle{empty}%
\begin{exercise}%
\textbf{test }%
\begin{enumerate}[label=\textbf{\Alph*. }]%
\item%
test%
\item%
test%
\item%
test%
\item%
test%
\end{enumerate}%
\end{exercise}%
\end{itemize}%
\newpage%
\thispagestyle{empty}%
test : MATHÉMATIQUE $\qquad \qquad \qquad \qquad \qquad \qquad \qquad \qquad$ EL HASSNAOUI Akram%
\begin{flushright}%
\begin{tabular}{|l|}%
\hline%
 \\%
\thispagestyle{empty}%
Identifiant: $\quad$ {\Large 9~0~2~3~4~}%
 \\%
\hline%
\end{tabular}%
\end{flushright}%
\begin{center}%
\begin{tabular}{| l l l l l |}%
\hline%
 & & & & \\%
Question 1\qquad \qquad\ & & & & \\%
 & A $\qquad \square \qquad$ & B $\qquad \square \qquad$ & C $\qquad \square \qquad$ & D $\qquad \square \qquad$ \\ %
 & & & &  \\%
\hline%
 & & & &  \\%
Choix 2 & A $\qquad \square \qquad$ & B $\qquad \square \qquad$ & C $\qquad \square \qquad$ & D $\qquad \square \qquad$ \\ %
 & & & &  \\%
\hline%
\end{tabular}%
\\ \vskip3mm%
\thispagestyle{empty}%
\end{center}%
\newpage%
\thispagestyle{empty}%
\vskip-40mm	\includegraphics[scale=0.5]{logo.png} \\%
 \begin{flushright}  \vskip-20mm   Professeur: Hatim Naqos\vskip15mm  \end{flushright}%
JM3%
 \begin{flushright}  \vskip-7mm2024-05-22 \end{flushright}%
\begin{center}   \begin{Large}test : MATHÉMATIQUE\end{Large} \end{center}%
Durée : 2h%
 \begin{center} { \large CONSIGNES SPÉCIFIQUES } \\\ Lisez soigneusement les consignes ci-dessous afin de réussir au mieux cette épreuve : \end{center} %
\begin{itemize}%
\item%
L'usage de la calculatrice ou de tout autre appareil électronique est interdit.%
\item%
Aucun autre document que ce sujet et sa grille réponse n'est autorisé.%
\item%
Pour chacune des questions, indiquez sur la feuille de réponses ci-jointes, si les affirmations A, B, C et D sont (\textbf{V}) vraies ou (\textbf{F}) fausses en faisant une croix dans la colonne correspondant à votre choix. Vous ne pouvez pas faire de ratures. En cas d'erreur, utilisez la deuxième colonne de réponse. Si la deuxième colonne comporte au moins une réponse, la première colonne ne sera pas corrigée, c'est la deuxième qui sera prise en considération.%
\item%
Chaque réponse exacte est gratifiée de 2 points, tandis que chaque réponse fausse est pénalisée par -1 point. \\ 	Parmi les quatre propositions de chacune des questions \textbf{de 1 à 1}, une seule est vraie, les autres sont fausses. \\ 	Par exemple : Pour indiquer que l'affirmation $B$ est Vraie, cocher les cases comme suit:  \\ \begin{center}	\includegraphics[scale=0.8]{reponses.png} \end{center}%
\thispagestyle{empty}%
\begin{exercise}%
\textbf{test }%
\begin{enumerate}[label=\textbf{\Alph*. }]%
\item%
test%
\item%
test%
\item%
test%
\item%
test%
\end{enumerate}%
\end{exercise}%
\end{itemize}%
\newpage%
\thispagestyle{empty}%
test : MATHÉMATIQUE $\qquad \qquad \qquad \qquad \qquad \qquad \qquad \qquad$ JBILOU Imane%
\begin{flushright}%
\begin{tabular}{|l|}%
\hline%
 \\%
\thispagestyle{empty}%
Identifiant: $\quad$ {\Large 1~3~4~5~6~}%
 \\%
\hline%
\end{tabular}%
\end{flushright}%
\begin{center}%
\begin{tabular}{| l l l l l |}%
\hline%
 & & & & \\%
Question 1\qquad \qquad\ & & & & \\%
 & A $\qquad \square \qquad$ & B $\qquad \square \qquad$ & C $\qquad \square \qquad$ & D $\qquad \square \qquad$ \\ %
 & & & &  \\%
\hline%
 & & & &  \\%
Choix 2 & A $\qquad \square \qquad$ & B $\qquad \square \qquad$ & C $\qquad \square \qquad$ & D $\qquad \square \qquad$ \\ %
 & & & &  \\%
\hline%
\end{tabular}%
\\ \vskip3mm%
\thispagestyle{empty}%
\end{center}%
\newpage%
\thispagestyle{empty}%
\vskip-40mm	\includegraphics[scale=0.5]{logo.png} \\%
 \begin{flushright}  \vskip-20mm   Professeur: Hatim Naqos\vskip15mm  \end{flushright}%
JM3%
 \begin{flushright}  \vskip-7mm2024-05-22 \end{flushright}%
\begin{center}   \begin{Large}test : MATHÉMATIQUE\end{Large} \end{center}%
Durée : 2h%
 \begin{center} { \large CONSIGNES SPÉCIFIQUES } \\\ Lisez soigneusement les consignes ci-dessous afin de réussir au mieux cette épreuve : \end{center} %
\begin{itemize}%
\item%
L'usage de la calculatrice ou de tout autre appareil électronique est interdit.%
\item%
Aucun autre document que ce sujet et sa grille réponse n'est autorisé.%
\item%
Pour chacune des questions, indiquez sur la feuille de réponses ci-jointes, si les affirmations A, B, C et D sont (\textbf{V}) vraies ou (\textbf{F}) fausses en faisant une croix dans la colonne correspondant à votre choix. Vous ne pouvez pas faire de ratures. En cas d'erreur, utilisez la deuxième colonne de réponse. Si la deuxième colonne comporte au moins une réponse, la première colonne ne sera pas corrigée, c'est la deuxième qui sera prise en considération.%
\item%
Chaque réponse exacte est gratifiée de 2 points, tandis que chaque réponse fausse est pénalisée par -1 point. \\ 	Parmi les quatre propositions de chacune des questions \textbf{de 1 à 1}, une seule est vraie, les autres sont fausses. \\ 	Par exemple : Pour indiquer que l'affirmation $B$ est Vraie, cocher les cases comme suit:  \\ \begin{center}	\includegraphics[scale=0.8]{reponses.png} \end{center}%
\thispagestyle{empty}%
\begin{exercise}%
\textbf{test }%
\begin{enumerate}[label=\textbf{\Alph*. }]%
\item%
test%
\item%
test%
\item%
test%
\item%
test%
\end{enumerate}%
\end{exercise}%
\end{itemize}%
\newpage%
\thispagestyle{empty}%
test : MATHÉMATIQUE $\qquad \qquad \qquad \qquad \qquad \qquad \qquad \qquad$ KHARBACH Younes%
\begin{flushright}%
\begin{tabular}{|l|}%
\hline%
 \\%
\thispagestyle{empty}%
Identifiant: $\quad$ {\Large 2~4~5~6~7~}%
 \\%
\hline%
\end{tabular}%
\end{flushright}%
\begin{center}%
\begin{tabular}{| l l l l l |}%
\hline%
 & & & & \\%
Question 1\qquad \qquad\ & & & & \\%
 & A $\qquad \square \qquad$ & B $\qquad \square \qquad$ & C $\qquad \square \qquad$ & D $\qquad \square \qquad$ \\ %
 & & & &  \\%
\hline%
 & & & &  \\%
Choix 2 & A $\qquad \square \qquad$ & B $\qquad \square \qquad$ & C $\qquad \square \qquad$ & D $\qquad \square \qquad$ \\ %
 & & & &  \\%
\hline%
\end{tabular}%
\\ \vskip3mm%
\thispagestyle{empty}%
\end{center}%
\newpage%
\thispagestyle{empty}%
\vskip-40mm	\includegraphics[scale=0.5]{logo.png} \\%
 \begin{flushright}  \vskip-20mm   Professeur: Hatim Naqos\vskip15mm  \end{flushright}%
JM3%
 \begin{flushright}  \vskip-7mm2024-05-22 \end{flushright}%
\begin{center}   \begin{Large}test : MATHÉMATIQUE\end{Large} \end{center}%
Durée : 2h%
 \begin{center} { \large CONSIGNES SPÉCIFIQUES } \\\ Lisez soigneusement les consignes ci-dessous afin de réussir au mieux cette épreuve : \end{center} %
\begin{itemize}%
\item%
L'usage de la calculatrice ou de tout autre appareil électronique est interdit.%
\item%
Aucun autre document que ce sujet et sa grille réponse n'est autorisé.%
\item%
Pour chacune des questions, indiquez sur la feuille de réponses ci-jointes, si les affirmations A, B, C et D sont (\textbf{V}) vraies ou (\textbf{F}) fausses en faisant une croix dans la colonne correspondant à votre choix. Vous ne pouvez pas faire de ratures. En cas d'erreur, utilisez la deuxième colonne de réponse. Si la deuxième colonne comporte au moins une réponse, la première colonne ne sera pas corrigée, c'est la deuxième qui sera prise en considération.%
\item%
Chaque réponse exacte est gratifiée de 2 points, tandis que chaque réponse fausse est pénalisée par -1 point. \\ 	Parmi les quatre propositions de chacune des questions \textbf{de 1 à 1}, une seule est vraie, les autres sont fausses. \\ 	Par exemple : Pour indiquer que l'affirmation $B$ est Vraie, cocher les cases comme suit:  \\ \begin{center}	\includegraphics[scale=0.8]{reponses.png} \end{center}%
\thispagestyle{empty}%
\begin{exercise}%
\textbf{test }%
\begin{enumerate}[label=\textbf{\Alph*. }]%
\item%
test%
\item%
test%
\item%
test%
\item%
test%
\end{enumerate}%
\end{exercise}%
\end{itemize}%
\newpage%
\thispagestyle{empty}%
test : MATHÉMATIQUE $\qquad \qquad \qquad \qquad \qquad \qquad \qquad \qquad$ LEKHBIOUI Hamza%
\begin{flushright}%
\begin{tabular}{|l|}%
\hline%
 \\%
\thispagestyle{empty}%
Identifiant: $\quad$ {\Large 3~5~6~7~8~}%
 \\%
\hline%
\end{tabular}%
\end{flushright}%
\begin{center}%
\begin{tabular}{| l l l l l |}%
\hline%
 & & & & \\%
Question 1\qquad \qquad\ & & & & \\%
 & A $\qquad \square \qquad$ & B $\qquad \square \qquad$ & C $\qquad \square \qquad$ & D $\qquad \square \qquad$ \\ %
 & & & &  \\%
\hline%
 & & & &  \\%
Choix 2 & A $\qquad \square \qquad$ & B $\qquad \square \qquad$ & C $\qquad \square \qquad$ & D $\qquad \square \qquad$ \\ %
 & & & &  \\%
\hline%
\end{tabular}%
\\ \vskip3mm%
\thispagestyle{empty}%
\end{center}%
\newpage%
\thispagestyle{empty}%
\vskip-40mm	\includegraphics[scale=0.5]{logo.png} \\%
 \begin{flushright}  \vskip-20mm   Professeur: Hatim Naqos\vskip15mm  \end{flushright}%
JM3%
 \begin{flushright}  \vskip-7mm2024-05-22 \end{flushright}%
\begin{center}   \begin{Large}test : MATHÉMATIQUE\end{Large} \end{center}%
Durée : 2h%
 \begin{center} { \large CONSIGNES SPÉCIFIQUES } \\\ Lisez soigneusement les consignes ci-dessous afin de réussir au mieux cette épreuve : \end{center} %
\begin{itemize}%
\item%
L'usage de la calculatrice ou de tout autre appareil électronique est interdit.%
\item%
Aucun autre document que ce sujet et sa grille réponse n'est autorisé.%
\item%
Pour chacune des questions, indiquez sur la feuille de réponses ci-jointes, si les affirmations A, B, C et D sont (\textbf{V}) vraies ou (\textbf{F}) fausses en faisant une croix dans la colonne correspondant à votre choix. Vous ne pouvez pas faire de ratures. En cas d'erreur, utilisez la deuxième colonne de réponse. Si la deuxième colonne comporte au moins une réponse, la première colonne ne sera pas corrigée, c'est la deuxième qui sera prise en considération.%
\item%
Chaque réponse exacte est gratifiée de 2 points, tandis que chaque réponse fausse est pénalisée par -1 point. \\ 	Parmi les quatre propositions de chacune des questions \textbf{de 1 à 1}, une seule est vraie, les autres sont fausses. \\ 	Par exemple : Pour indiquer que l'affirmation $B$ est Vraie, cocher les cases comme suit:  \\ \begin{center}	\includegraphics[scale=0.8]{reponses.png} \end{center}%
\thispagestyle{empty}%
\begin{exercise}%
\textbf{test }%
\begin{enumerate}[label=\textbf{\Alph*. }]%
\item%
test%
\item%
test%
\item%
test%
\item%
test%
\end{enumerate}%
\end{exercise}%
\end{itemize}%
\newpage%
\thispagestyle{empty}%
test : MATHÉMATIQUE $\qquad \qquad \qquad \qquad \qquad \qquad \qquad \qquad$ MOUMEN Abderrahmane%
\begin{flushright}%
\begin{tabular}{|l|}%
\hline%
 \\%
\thispagestyle{empty}%
Identifiant: $\quad$ {\Large 4~6~7~8~9~}%
 \\%
\hline%
\end{tabular}%
\end{flushright}%
\begin{center}%
\begin{tabular}{| l l l l l |}%
\hline%
 & & & & \\%
Question 1\qquad \qquad\ & & & & \\%
 & A $\qquad \square \qquad$ & B $\qquad \square \qquad$ & C $\qquad \square \qquad$ & D $\qquad \square \qquad$ \\ %
 & & & &  \\%
\hline%
 & & & &  \\%
Choix 2 & A $\qquad \square \qquad$ & B $\qquad \square \qquad$ & C $\qquad \square \qquad$ & D $\qquad \square \qquad$ \\ %
 & & & &  \\%
\hline%
\end{tabular}%
\\ \vskip3mm%
\thispagestyle{empty}%
\end{center}%
\newpage%
\thispagestyle{empty}%
\vskip-40mm	\includegraphics[scale=0.5]{logo.png} \\%
 \begin{flushright}  \vskip-20mm   Professeur: Hatim Naqos\vskip15mm  \end{flushright}%
JM3%
 \begin{flushright}  \vskip-7mm2024-05-22 \end{flushright}%
\begin{center}   \begin{Large}test : MATHÉMATIQUE\end{Large} \end{center}%
Durée : 2h%
 \begin{center} { \large CONSIGNES SPÉCIFIQUES } \\\ Lisez soigneusement les consignes ci-dessous afin de réussir au mieux cette épreuve : \end{center} %
\begin{itemize}%
\item%
L'usage de la calculatrice ou de tout autre appareil électronique est interdit.%
\item%
Aucun autre document que ce sujet et sa grille réponse n'est autorisé.%
\item%
Pour chacune des questions, indiquez sur la feuille de réponses ci-jointes, si les affirmations A, B, C et D sont (\textbf{V}) vraies ou (\textbf{F}) fausses en faisant une croix dans la colonne correspondant à votre choix. Vous ne pouvez pas faire de ratures. En cas d'erreur, utilisez la deuxième colonne de réponse. Si la deuxième colonne comporte au moins une réponse, la première colonne ne sera pas corrigée, c'est la deuxième qui sera prise en considération.%
\item%
Chaque réponse exacte est gratifiée de 2 points, tandis que chaque réponse fausse est pénalisée par -1 point. \\ 	Parmi les quatre propositions de chacune des questions \textbf{de 1 à 1}, une seule est vraie, les autres sont fausses. \\ 	Par exemple : Pour indiquer que l'affirmation $B$ est Vraie, cocher les cases comme suit:  \\ \begin{center}	\includegraphics[scale=0.8]{reponses.png} \end{center}%
\thispagestyle{empty}%
\begin{exercise}%
\textbf{test }%
\begin{enumerate}[label=\textbf{\Alph*. }]%
\item%
test%
\item%
test%
\item%
test%
\item%
test%
\end{enumerate}%
\end{exercise}%
\end{itemize}%
\newpage%
\thispagestyle{empty}%
test : MATHÉMATIQUE $\qquad \qquad \qquad \qquad \qquad \qquad \qquad \qquad$ MOUSSAIF Mohammed Amine%
\begin{flushright}%
\begin{tabular}{|l|}%
\hline%
 \\%
\thispagestyle{empty}%
Identifiant: $\quad$ {\Large 5~7~8~9~0~}%
 \\%
\hline%
\end{tabular}%
\end{flushright}%
\begin{center}%
\begin{tabular}{| l l l l l |}%
\hline%
 & & & & \\%
Question 1\qquad \qquad\ & & & & \\%
 & A $\qquad \square \qquad$ & B $\qquad \square \qquad$ & C $\qquad \square \qquad$ & D $\qquad \square \qquad$ \\ %
 & & & &  \\%
\hline%
 & & & &  \\%
Choix 2 & A $\qquad \square \qquad$ & B $\qquad \square \qquad$ & C $\qquad \square \qquad$ & D $\qquad \square \qquad$ \\ %
 & & & &  \\%
\hline%
\end{tabular}%
\\ \vskip3mm%
\thispagestyle{empty}%
\end{center}%
\newpage%
\thispagestyle{empty}%
\vskip-40mm	\includegraphics[scale=0.5]{logo.png} \\%
 \begin{flushright}  \vskip-20mm   Professeur: Hatim Naqos\vskip15mm  \end{flushright}%
JM3%
 \begin{flushright}  \vskip-7mm2024-05-22 \end{flushright}%
\begin{center}   \begin{Large}test : MATHÉMATIQUE\end{Large} \end{center}%
Durée : 2h%
 \begin{center} { \large CONSIGNES SPÉCIFIQUES } \\\ Lisez soigneusement les consignes ci-dessous afin de réussir au mieux cette épreuve : \end{center} %
\begin{itemize}%
\item%
L'usage de la calculatrice ou de tout autre appareil électronique est interdit.%
\item%
Aucun autre document que ce sujet et sa grille réponse n'est autorisé.%
\item%
Pour chacune des questions, indiquez sur la feuille de réponses ci-jointes, si les affirmations A, B, C et D sont (\textbf{V}) vraies ou (\textbf{F}) fausses en faisant une croix dans la colonne correspondant à votre choix. Vous ne pouvez pas faire de ratures. En cas d'erreur, utilisez la deuxième colonne de réponse. Si la deuxième colonne comporte au moins une réponse, la première colonne ne sera pas corrigée, c'est la deuxième qui sera prise en considération.%
\item%
Chaque réponse exacte est gratifiée de 2 points, tandis que chaque réponse fausse est pénalisée par -1 point. \\ 	Parmi les quatre propositions de chacune des questions \textbf{de 1 à 1}, une seule est vraie, les autres sont fausses. \\ 	Par exemple : Pour indiquer que l'affirmation $B$ est Vraie, cocher les cases comme suit:  \\ \begin{center}	\includegraphics[scale=0.8]{reponses.png} \end{center}%
\thispagestyle{empty}%
\begin{exercise}%
\textbf{test }%
\begin{enumerate}[label=\textbf{\Alph*. }]%
\item%
test%
\item%
test%
\item%
test%
\item%
test%
\end{enumerate}%
\end{exercise}%
\end{itemize}%
\newpage%
\thispagestyle{empty}%
test : MATHÉMATIQUE $\qquad \qquad \qquad \qquad \qquad \qquad \qquad \qquad$ NASIR ELHAK Mohamed Walid%
\begin{flushright}%
\begin{tabular}{|l|}%
\hline%
 \\%
\thispagestyle{empty}%
Identifiant: $\quad$ {\Large 6~8~9~0~1~}%
 \\%
\hline%
\end{tabular}%
\end{flushright}%
\begin{center}%
\begin{tabular}{| l l l l l |}%
\hline%
 & & & & \\%
Question 1\qquad \qquad\ & & & & \\%
 & A $\qquad \square \qquad$ & B $\qquad \square \qquad$ & C $\qquad \square \qquad$ & D $\qquad \square \qquad$ \\ %
 & & & &  \\%
\hline%
 & & & &  \\%
Choix 2 & A $\qquad \square \qquad$ & B $\qquad \square \qquad$ & C $\qquad \square \qquad$ & D $\qquad \square \qquad$ \\ %
 & & & &  \\%
\hline%
\end{tabular}%
\\ \vskip3mm%
\thispagestyle{empty}%
\end{center}%
\newpage%
\thispagestyle{empty}%
\vskip-40mm	\includegraphics[scale=0.5]{logo.png} \\%
 \begin{flushright}  \vskip-20mm   Professeur: Hatim Naqos\vskip15mm  \end{flushright}%
JM3%
 \begin{flushright}  \vskip-7mm2024-05-22 \end{flushright}%
\begin{center}   \begin{Large}test : MATHÉMATIQUE\end{Large} \end{center}%
Durée : 2h%
 \begin{center} { \large CONSIGNES SPÉCIFIQUES } \\\ Lisez soigneusement les consignes ci-dessous afin de réussir au mieux cette épreuve : \end{center} %
\begin{itemize}%
\item%
L'usage de la calculatrice ou de tout autre appareil électronique est interdit.%
\item%
Aucun autre document que ce sujet et sa grille réponse n'est autorisé.%
\item%
Pour chacune des questions, indiquez sur la feuille de réponses ci-jointes, si les affirmations A, B, C et D sont (\textbf{V}) vraies ou (\textbf{F}) fausses en faisant une croix dans la colonne correspondant à votre choix. Vous ne pouvez pas faire de ratures. En cas d'erreur, utilisez la deuxième colonne de réponse. Si la deuxième colonne comporte au moins une réponse, la première colonne ne sera pas corrigée, c'est la deuxième qui sera prise en considération.%
\item%
Chaque réponse exacte est gratifiée de 2 points, tandis que chaque réponse fausse est pénalisée par -1 point. \\ 	Parmi les quatre propositions de chacune des questions \textbf{de 1 à 1}, une seule est vraie, les autres sont fausses. \\ 	Par exemple : Pour indiquer que l'affirmation $B$ est Vraie, cocher les cases comme suit:  \\ \begin{center}	\includegraphics[scale=0.8]{reponses.png} \end{center}%
\thispagestyle{empty}%
\begin{exercise}%
\textbf{test }%
\begin{enumerate}[label=\textbf{\Alph*. }]%
\item%
test%
\item%
test%
\item%
test%
\item%
test%
\end{enumerate}%
\end{exercise}%
\end{itemize}%
\newpage%
\thispagestyle{empty}%
test : MATHÉMATIQUE $\qquad \qquad \qquad \qquad \qquad \qquad \qquad \qquad$ NATHAN JORLYN Mvomo%
\begin{flushright}%
\begin{tabular}{|l|}%
\hline%
 \\%
\thispagestyle{empty}%
Identifiant: $\quad$ {\Large 7~9~0~1~2~}%
 \\%
\hline%
\end{tabular}%
\end{flushright}%
\begin{center}%
\begin{tabular}{| l l l l l |}%
\hline%
 & & & & \\%
Question 1\qquad \qquad\ & & & & \\%
 & A $\qquad \square \qquad$ & B $\qquad \square \qquad$ & C $\qquad \square \qquad$ & D $\qquad \square \qquad$ \\ %
 & & & &  \\%
\hline%
 & & & &  \\%
Choix 2 & A $\qquad \square \qquad$ & B $\qquad \square \qquad$ & C $\qquad \square \qquad$ & D $\qquad \square \qquad$ \\ %
 & & & &  \\%
\hline%
\end{tabular}%
\\ \vskip3mm%
\thispagestyle{empty}%
\end{center}%
\newpage%
\thispagestyle{empty}%
\vskip-40mm	\includegraphics[scale=0.5]{logo.png} \\%
 \begin{flushright}  \vskip-20mm   Professeur: Hatim Naqos\vskip15mm  \end{flushright}%
JM3%
 \begin{flushright}  \vskip-7mm2024-05-22 \end{flushright}%
\begin{center}   \begin{Large}test : MATHÉMATIQUE\end{Large} \end{center}%
Durée : 2h%
 \begin{center} { \large CONSIGNES SPÉCIFIQUES } \\\ Lisez soigneusement les consignes ci-dessous afin de réussir au mieux cette épreuve : \end{center} %
\begin{itemize}%
\item%
L'usage de la calculatrice ou de tout autre appareil électronique est interdit.%
\item%
Aucun autre document que ce sujet et sa grille réponse n'est autorisé.%
\item%
Pour chacune des questions, indiquez sur la feuille de réponses ci-jointes, si les affirmations A, B, C et D sont (\textbf{V}) vraies ou (\textbf{F}) fausses en faisant une croix dans la colonne correspondant à votre choix. Vous ne pouvez pas faire de ratures. En cas d'erreur, utilisez la deuxième colonne de réponse. Si la deuxième colonne comporte au moins une réponse, la première colonne ne sera pas corrigée, c'est la deuxième qui sera prise en considération.%
\item%
Chaque réponse exacte est gratifiée de 2 points, tandis que chaque réponse fausse est pénalisée par -1 point. \\ 	Parmi les quatre propositions de chacune des questions \textbf{de 1 à 1}, une seule est vraie, les autres sont fausses. \\ 	Par exemple : Pour indiquer que l'affirmation $B$ est Vraie, cocher les cases comme suit:  \\ \begin{center}	\includegraphics[scale=0.8]{reponses.png} \end{center}%
\thispagestyle{empty}%
\begin{exercise}%
\textbf{test }%
\begin{enumerate}[label=\textbf{\Alph*. }]%
\item%
test%
\item%
test%
\item%
test%
\item%
test%
\end{enumerate}%
\end{exercise}%
\end{itemize}%
\newpage%
\thispagestyle{empty}%
test : MATHÉMATIQUE $\qquad \qquad \qquad \qquad \qquad \qquad \qquad \qquad$ SAHNOUNI Ilyas%
\begin{flushright}%
\begin{tabular}{|l|}%
\hline%
 \\%
\thispagestyle{empty}%
Identifiant: $\quad$ {\Large 8~1~2~3~4~}%
 \\%
\hline%
\end{tabular}%
\end{flushright}%
\begin{center}%
\begin{tabular}{| l l l l l |}%
\hline%
 & & & & \\%
Question 1\qquad \qquad\ & & & & \\%
 & A $\qquad \square \qquad$ & B $\qquad \square \qquad$ & C $\qquad \square \qquad$ & D $\qquad \square \qquad$ \\ %
 & & & &  \\%
\hline%
 & & & &  \\%
Choix 2 & A $\qquad \square \qquad$ & B $\qquad \square \qquad$ & C $\qquad \square \qquad$ & D $\qquad \square \qquad$ \\ %
 & & & &  \\%
\hline%
\end{tabular}%
\\ \vskip3mm%
\thispagestyle{empty}%
\end{center}%
\newpage%
\thispagestyle{empty}%
\vskip-40mm	\includegraphics[scale=0.5]{logo.png} \\%
 \begin{flushright}  \vskip-20mm   Professeur: Hatim Naqos\vskip15mm  \end{flushright}%
JM3%
 \begin{flushright}  \vskip-7mm2024-05-22 \end{flushright}%
\begin{center}   \begin{Large}test : MATHÉMATIQUE\end{Large} \end{center}%
Durée : 2h%
 \begin{center} { \large CONSIGNES SPÉCIFIQUES } \\\ Lisez soigneusement les consignes ci-dessous afin de réussir au mieux cette épreuve : \end{center} %
\begin{itemize}%
\item%
L'usage de la calculatrice ou de tout autre appareil électronique est interdit.%
\item%
Aucun autre document que ce sujet et sa grille réponse n'est autorisé.%
\item%
Pour chacune des questions, indiquez sur la feuille de réponses ci-jointes, si les affirmations A, B, C et D sont (\textbf{V}) vraies ou (\textbf{F}) fausses en faisant une croix dans la colonne correspondant à votre choix. Vous ne pouvez pas faire de ratures. En cas d'erreur, utilisez la deuxième colonne de réponse. Si la deuxième colonne comporte au moins une réponse, la première colonne ne sera pas corrigée, c'est la deuxième qui sera prise en considération.%
\item%
Chaque réponse exacte est gratifiée de 2 points, tandis que chaque réponse fausse est pénalisée par -1 point. \\ 	Parmi les quatre propositions de chacune des questions \textbf{de 1 à 1}, une seule est vraie, les autres sont fausses. \\ 	Par exemple : Pour indiquer que l'affirmation $B$ est Vraie, cocher les cases comme suit:  \\ \begin{center}	\includegraphics[scale=0.8]{reponses.png} \end{center}%
\thispagestyle{empty}%
\begin{exercise}%
\textbf{test }%
\begin{enumerate}[label=\textbf{\Alph*. }]%
\item%
test%
\item%
test%
\item%
test%
\item%
test%
\end{enumerate}%
\end{exercise}%
\end{itemize}%
\newpage%
\thispagestyle{empty}%
test : MATHÉMATIQUE $\qquad \qquad \qquad \qquad \qquad \qquad \qquad \qquad$ SALIFOU WAZAMA Wazama %
\begin{flushright}%
\begin{tabular}{|l|}%
\hline%
 \\%
\thispagestyle{empty}%
Identifiant: $\quad$ {\Large 9~2~3~4~5~}%
 \\%
\hline%
\end{tabular}%
\end{flushright}%
\begin{center}%
\begin{tabular}{| l l l l l |}%
\hline%
 & & & & \\%
Question 1\qquad \qquad\ & & & & \\%
 & A $\qquad \square \qquad$ & B $\qquad \square \qquad$ & C $\qquad \square \qquad$ & D $\qquad \square \qquad$ \\ %
 & & & &  \\%
\hline%
 & & & &  \\%
Choix 2 & A $\qquad \square \qquad$ & B $\qquad \square \qquad$ & C $\qquad \square \qquad$ & D $\qquad \square \qquad$ \\ %
 & & & &  \\%
\hline%
\end{tabular}%
\\ \vskip3mm%
\thispagestyle{empty}%
\end{center}%
\newpage%
\thispagestyle{empty}%
\vskip-40mm	\includegraphics[scale=0.5]{logo.png} \\%
 \begin{flushright}  \vskip-20mm   Professeur: Hatim Naqos\vskip15mm  \end{flushright}%
JM3%
 \begin{flushright}  \vskip-7mm2024-05-22 \end{flushright}%
\begin{center}   \begin{Large}test : MATHÉMATIQUE\end{Large} \end{center}%
Durée : 2h%
 \begin{center} { \large CONSIGNES SPÉCIFIQUES } \\\ Lisez soigneusement les consignes ci-dessous afin de réussir au mieux cette épreuve : \end{center} %
\begin{itemize}%
\item%
L'usage de la calculatrice ou de tout autre appareil électronique est interdit.%
\item%
Aucun autre document que ce sujet et sa grille réponse n'est autorisé.%
\item%
Pour chacune des questions, indiquez sur la feuille de réponses ci-jointes, si les affirmations A, B, C et D sont (\textbf{V}) vraies ou (\textbf{F}) fausses en faisant une croix dans la colonne correspondant à votre choix. Vous ne pouvez pas faire de ratures. En cas d'erreur, utilisez la deuxième colonne de réponse. Si la deuxième colonne comporte au moins une réponse, la première colonne ne sera pas corrigée, c'est la deuxième qui sera prise en considération.%
\item%
Chaque réponse exacte est gratifiée de 2 points, tandis que chaque réponse fausse est pénalisée par -1 point. \\ 	Parmi les quatre propositions de chacune des questions \textbf{de 1 à 1}, une seule est vraie, les autres sont fausses. \\ 	Par exemple : Pour indiquer que l'affirmation $B$ est Vraie, cocher les cases comme suit:  \\ \begin{center}	\includegraphics[scale=0.8]{reponses.png} \end{center}%
\thispagestyle{empty}%
\begin{exercise}%
\textbf{test }%
\begin{enumerate}[label=\textbf{\Alph*. }]%
\item%
test%
\item%
test%
\item%
test%
\item%
test%
\end{enumerate}%
\end{exercise}%
\end{itemize}%
\newpage%
\thispagestyle{empty}%
test : MATHÉMATIQUE $\qquad \qquad \qquad \qquad \qquad \qquad \qquad \qquad$ TAZI CHIBI Ayoub%
\begin{flushright}%
\begin{tabular}{|l|}%
\hline%
 \\%
\thispagestyle{empty}%
Identifiant: $\quad$ {\Large 1~0~4~5~6~}%
 \\%
\hline%
\end{tabular}%
\end{flushright}%
\begin{center}%
\begin{tabular}{| l l l l l |}%
\hline%
 & & & & \\%
Question 1\qquad \qquad\ & & & & \\%
 & A $\qquad \square \qquad$ & B $\qquad \square \qquad$ & C $\qquad \square \qquad$ & D $\qquad \square \qquad$ \\ %
 & & & &  \\%
\hline%
 & & & &  \\%
Choix 2 & A $\qquad \square \qquad$ & B $\qquad \square \qquad$ & C $\qquad \square \qquad$ & D $\qquad \square \qquad$ \\ %
 & & & &  \\%
\hline%
\end{tabular}%
\\ \vskip3mm%
\thispagestyle{empty}%
\end{center}%
\newpage%
\thispagestyle{empty}%
\vskip-40mm	\includegraphics[scale=0.5]{logo.png} \\%
 \begin{flushright}  \vskip-20mm   Professeur: Hatim Naqos\vskip15mm  \end{flushright}%
JM3%
 \begin{flushright}  \vskip-7mm2024-05-22 \end{flushright}%
\begin{center}   \begin{Large}test : MATHÉMATIQUE\end{Large} \end{center}%
Durée : 2h%
 \begin{center} { \large CONSIGNES SPÉCIFIQUES } \\\ Lisez soigneusement les consignes ci-dessous afin de réussir au mieux cette épreuve : \end{center} %
\begin{itemize}%
\item%
L'usage de la calculatrice ou de tout autre appareil électronique est interdit.%
\item%
Aucun autre document que ce sujet et sa grille réponse n'est autorisé.%
\item%
Pour chacune des questions, indiquez sur la feuille de réponses ci-jointes, si les affirmations A, B, C et D sont (\textbf{V}) vraies ou (\textbf{F}) fausses en faisant une croix dans la colonne correspondant à votre choix. Vous ne pouvez pas faire de ratures. En cas d'erreur, utilisez la deuxième colonne de réponse. Si la deuxième colonne comporte au moins une réponse, la première colonne ne sera pas corrigée, c'est la deuxième qui sera prise en considération.%
\item%
Chaque réponse exacte est gratifiée de 2 points, tandis que chaque réponse fausse est pénalisée par -1 point. \\ 	Parmi les quatre propositions de chacune des questions \textbf{de 1 à 1}, une seule est vraie, les autres sont fausses. \\ 	Par exemple : Pour indiquer que l'affirmation $B$ est Vraie, cocher les cases comme suit:  \\ \begin{center}	\includegraphics[scale=0.8]{reponses.png} \end{center}%
\thispagestyle{empty}%
\begin{exercise}%
\textbf{test }%
\begin{enumerate}[label=\textbf{\Alph*. }]%
\item%
test%
\item%
test%
\item%
test%
\item%
test%
\end{enumerate}%
\end{exercise}%
\end{itemize}%
\newpage%
\thispagestyle{empty}%
test : MATHÉMATIQUE $\qquad \qquad \qquad \qquad \qquad \qquad \qquad \qquad$ ZAKI Ilias %
\begin{flushright}%
\begin{tabular}{|l|}%
\hline%
 \\%
\thispagestyle{empty}%
Identifiant: $\quad$ {\Large 2~1~5~6~7~}%
 \\%
\hline%
\end{tabular}%
\end{flushright}%
\begin{center}%
\begin{tabular}{| l l l l l |}%
\hline%
 & & & & \\%
Question 1\qquad \qquad\ & & & & \\%
 & A $\qquad \square \qquad$ & B $\qquad \square \qquad$ & C $\qquad \square \qquad$ & D $\qquad \square \qquad$ \\ %
 & & & &  \\%
\hline%
 & & & &  \\%
Choix 2 & A $\qquad \square \qquad$ & B $\qquad \square \qquad$ & C $\qquad \square \qquad$ & D $\qquad \square \qquad$ \\ %
 & & & &  \\%
\hline%
\end{tabular}%
\\ \vskip3mm%
\thispagestyle{empty}%
\end{center}%
\end{document}